\chapter{Darwin Library}
\label{sec:darwin_lib}

\cleanchapterquote{Innovation distinguishes between a leader and a follower.}{Steve Jobs}{(CEO Apple Inc.)}

\section{Structure}
\label{sec:system:sec1}

Talk about how the library is organized.
Mainly, we have two divisions:

\begin{itemize}
  \item Potential
  \item Graphs
  \item Inference
\end{itemize}

A set of tools for potential manipulations and a set of tools for graph manipulation

\section{Features}
\label{sec:system:sec2}

Which operations the library can do.
Example:

\begin{itemize}
  \item Variable elimination
  \item d-Separation
  \item Convert BN to MN
  \item Triangulation with heuristics
  \item graph manipulation uses NetworkX.
\end{itemize}

How things are implemented.
For example: Koller's algorithm for potential implementation.

\section{Usage}
\label{sec:system:sec3}

Main classes and how to use them.
Example:

\begin{itemize}
  \item Class Potential
  \item Class Graph With Potential
  \item Class BN
  \item Class MN
\end{itemize}
