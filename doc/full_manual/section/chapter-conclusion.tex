\chapter{Conclusion}
\label{sec:conclusion}

We have suggested Darwinian Network Library as new framework for modeling and inference in DNs.
DN library is a implementation of the basic operations for adaptation and evolution in DNs.
It is implemented with the Python programming language and the basic data structures are provided by Darwin library.
It provides salient features as test unity and a tutorials with IPython Notebook.
The main advantage of the DN library is to apply novel ideas and techniques of DNs.
DN library also is great for teaching as well for quick prototyping or testing.


DN library is based upon the novel library Darwin, a written in Python for BN modelling and inference.
Darwin library has two main categories called potentials and graph manipulations.
The former is a data structure for modelling a probability table.
The latter maps nodes in the graph to a list of potentials


We have established in Theorem \ref{the:adap_eq_t_adap} that t-adaptation is equivalent to adaptation.
Thus, t-adaptation will always yield the same result as adaptation, but perhaps with less computation.














%The founder of BNs writes that perhaps d-separation had the greatest immediate impact of \cite{pearl86}. He also writes later \cite{pearl09} that many have had difficulties in understanding d-separation.
%One of the reasons this is is because d-separation is sometimes counter-intuitive, namely, a path that necessarily traverses the conditioning set can be considered as not being blocked.
%In this paper, we have suggested a pruning step on the DAG so that blocking works in an intuitive way.
%
%In addition, we have simplified m-separation, which is an equivalent method to d-separation.
%We have explicitly demonstrated how m-separation can prune edges unnecessarily.
%That is, classical separation will give the same result whether or not these edges are pruned.
%Moreover, we have also shown that m-separation can unnecessarily add edges to the undirected graph.
%Once again the addition of these edges does not affect the result of classical separation.
%We have suggested the minimum edges that need to be added to ensure correctness.