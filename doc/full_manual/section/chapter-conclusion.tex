\chapter{Conclusion}
\label{sec:conclusion}

We propose Darwinian Network Library as new framework for modeling and inference in DNs.
DN library is a implementation of the basic operations for adaptation and evolution in DNs.
It is implemented with the Python programming language and the basic data structures are provided by Darwin library.
It provides salient features as test unity and a tutorials with IPython Notebook.
The main advantage of the DN library is to apply novel ideas and techniques of DNs.
DN library also is great for teaching as well for quick prototyping or testing.


DN library is based upon the novel library Darwin, a written in Python for BN modeling and inference.
Darwin library has two main categories called potentials and graph manipulations.
The former is a data structure for modeling a probability table.
The latter maps nodes in the graph to a list of potentials.


We have established in Chapter \ref{sec:dn_lib} features of DN library given the manipulation tools from Darwin.
We have shown that DN library is an intuitive framework for working with DNs.
For instance, to a set of potentials, that is, a DN, one can utilize the class \emph{Population} that together form the other class \emph{DarwinianNetwork}.
Another salient feature is the set of procedures for drawing Populations and DarwinianNetworks.
It allows the user to see how looks like a population that has just been created - or even the whole DN!


This project proposes the implementation of a DN library.
In summary, we have stablished three main advantages of using DN library:
\begin{inparaenum}[(i)]
\item due its simple approach given the Python implementation, it has an expressive language that facilitate its usage;
\item it has interactive visual tools to compute and draw DNs which enables users a interesting interactivity with the library;
\item it is a great tool for learning and its operations is sound as assure by unit tests;
\item and all source code is available online on \emph{GitHub} through the webpage:
\end{inparaenum}
\begin{center}
\url{https://github.com/Darwinian-Networks}
\end{center}
Finally, with DN library, DNs can be applyied as a simple and yet remarkably robust tool to simplify reasoning with BNs.
